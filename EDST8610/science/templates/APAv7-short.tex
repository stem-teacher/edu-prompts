\documentclass[jou]{apa7}

\usepackage[american]{babel}

\usepackage{csquotes}
\usepackage[style=apa,sortcites=true,sorting=nyt,backend=biber]{biblatex}
\DeclareLanguageMapping{american}{american-apa}
\addbibresource{bibliography.bib}

\title{Response Preparation for HSC Investigating Science}

\author{Philip Haynes}
\affiliation{Gosford High School}
\authornote{
   \addORCIDlink{Philip Haynes}{https://orcid.org/0009-0003-6377-4791}

  Correspondence concerning this article should be addressed to Philip Haynes, Science Department, Gosford High School, Racecourse Road, Gosford, NSW, 2250  E-mail: philip.haynes4@det.nsw.edu.au}

\leftheader{Haynes}

\abstract{This paper provides .}

\keywords{Investigating Science, Exam Preparation}

\begin{document}
\maketitle
We begin with \textcite{Shotton1989}.  We can also cite this work in
parenthesis, like this: \parencite{Shotton1989}.

A three-author paper \parencite[e.g.,][]{Lassen2006} lists all
three authors for the first citation, then only the first author
on all subsequent citations \parencite{Lassen2006}.

Note the use of five heading levels throughout this demonstration
Method section.

\section{Method}
\subsection{Participants}
We had a lot of people in this study.

\subsection{Materials}
Several materials were used for this project.  Some of themwere
already created for prior research.

\subsubsection{Paper-and-Pencil Instrument}
We used an instrument that we found to be highly successful.

\paragraph{Reliability}
The reliability of this instrument is extraordinary.

\paragraph{Validity}
We now discuss the validity of our instrument.

\subparagraph{Face validity} The face validity is exceptionally
strong.  Everyone should be impressed.

\subparagraph{Construct validity} Also very strong.

\subsection{Design}
This section describes the study's design.

\subsection{Procedure}
The procedure was fairly straightforward, yet required
attention to detail.

\section{Results}
Table \ref{tab:ComplexTable} contains some sample data.  Our
statistical prowess in analyzing these data is unmatched.

\begin{table}[htbp]
  \vspace*{2em}
  \begin{threeparttable}
    \caption{A Complex Table}
    \label{tab:ComplexTable}
    \begin{tabular}{@{}lrrr@{}}         \toprule
    Distribution type  & \multicolumn{2}{l}{Percentage of} & Total number   \\
                       & \multicolumn{2}{l}{targets with}  & of trials per  \\
                       & \multicolumn{2}{l}{segment in}    & participant    \\ \cmidrule(r){2-3}
                                    &  Onset  &  Coda            &          \\ \midrule
    Categorical -- onset\tabfnm{a}  &    100  &     0            &  196     \\
    Probabilistic                   &     80  &    20\tabfnm{*}  &  200     \\
    Categorical -- coda\tabfnm{b}   &      0  &   100\tabfnm{*}  &  196     \\ \midrule
    \end{tabular}
    \begin{tablenotes}[para,flushleft]
        {\small
            \textit{Note.} All data are approximate.

            \tabfnt{a}Categorical may be onset.
            \tabfnt{b}Categorical may also be coda.

            \tabfnt{*}\textit{p} < .05.
            \tabfnt{**}\textit{p} < .01.
         }
    \end{tablenotes}
  \end{threeparttable}
\end{table}

\section{Discussion}
This is a lengthy and erudite discussion.  It demonstrates amazing
skill in interpreting the results for the masses.

\printbibliography

\end{document}
