\documentclass[xcolor=svgnames]{beamer}
\usepackage[utf8]{inputenc}
\usepackage{xcolor}
\usepackage{booktabs, comment}
\usepackage{pgfpages}
\usepackage{csquotes}
\usepackage{amsmath}
\usepackage{tikz}
\usetheme{Madrid}

% COLORS
\definecolor{mqred}{RGB}{166, 25, 46}
\definecolor{mqdeepred}{RGB}{118, 35, 47}
\definecolor{mqgray}{RGB}{55, 58, 54}
\definecolor{mqlightgray}{RGB}{237, 235, 229}
\definecolor{mqmagenta}{RGB}{198, 0, 126}
\usecolortheme[named=mqred]{structure}
\setbeamercolor{title in head/foot}{bg=mqlightgray, fg=mqgray}
\setbeamercolor{author in head/foot}{bg=mqdeepred}
\setbeamercolor{page number in head/foot}{bg=mqdeepred, fg=mqlightgray}

% FOOTNOTE ARRANGEMENTS
\makeatletter
\setbeamertemplate{footline}{
  \leavevmode%
  \hbox{%
  \begin{beamercolorbox}[wd=.5\paperwidth,ht=2.25ex,dp=1ex,center]{author in head/foot}%
    \usebeamerfont{author in head/foot}\insertshortauthor\expandafter\ifblank\expandafter{\beamer@shortinstitute}{}{~~(\insertshortinstitute)}
  \end{beamercolorbox}%
  \begin{beamercolorbox}[wd=.4\paperwidth,ht=2.25ex,dp=1ex,center]{title in head/foot}%
    \usebeamerfont{title in head/foot}\insertshorttitle
  \end{beamercolorbox}%
  \begin{beamercolorbox}[wd=.1\paperwidth,ht=2.25ex,dp=1ex,center]{page number in head/foot}%
    \usebeamerfont{page number in head/foot}\insertframenumber{} / \inserttotalframenumber
  \end{beamercolorbox}}%
  \vskip0pt%
}
\makeatother
\beamertemplatenavigationsymbolsempty

% TITLE, AUTHORS, INSTITUTE, DATE
\title[Equations of Motion]{Equations of Motion: Motion Graphs and Equations}
\author[Philip Haynes]{Philip Haynes}
\institute[Gosford High School]{Gosford High School}
\date{10/01/2025}

% LOGO
\titlegraphic{\includegraphics[height=2.5cm]{logo.jpg}} % Change the logo path as needed

\begin{document}

\begin{frame}
    \titlepage
\end{frame}

\begin{frame}{Outline}
    \tableofcontents
\end{frame}

\section{Introduction}
\begin{frame}{Introduction to Motion Concepts}
    \begin{itemize}
        \item Definitions: displacement, velocity, acceleration.
        \item Real-world examples: falling objects, moving vehicles.
        \item Feynman approach: starting with simple points and building complexity.
    \end{itemize}
\end{frame}

\section{Motion Graphs}
\begin{frame}{Understanding Motion Graphs}
    \begin{itemize}
        \item Displacement--Time Graphs: What they show.
        \item Velocity--Time Graphs: Deriving velocity from displacement.
        \item Relationship between graphs and equations.
        \item Interactive demonstrations (refer to interactive session).
    \end{itemize}
\end{frame}

\section{Equations of Motion}
\begin{frame}{Derivation of Equations}
    \begin{itemize}
        \item Start with constant acceleration: $g=9.8$ m/s$^2$.
        \item Equations:
        \begin{itemize}
            \item $v = u + at$
            \item $s = ut + \frac{1}{2}at^2$
            \item $v^2 = u^2 + 2as$
        \end{itemize}
        \item Use Feynman's method: taking limits to define instantaneous values.
    \end{itemize}
\end{frame}

\section{Worked Examples}
\begin{frame}{Worked Example: Free Fall}
    \begin{itemize}
        \item An object starts from rest ($u=0$) and falls under gravity.
        \item Calculate velocity after 5 seconds: $v = 0 + 9.8(5) = 49$ m/s.
        \item Calculate displacement: $s = 0 + \frac{1}{2}(9.8)(5^2) = 122.5$ m.
        \item Discuss graph representations.
    \end{itemize}
\end{frame}

\section{Practice Problems and Review}
\begin{frame}{Practice Problems}
    \begin{itemize}
        \item Solve similar problems in groups.
        \item Use interactive tools to verify answers.
        \item Immediate feedback through class discussion.
    \end{itemize}
\end{frame}

\begin{frame}{Summary and Key Takeaways}
    \begin{itemize}
        \item Motion graphs illustrate changes in displacement and velocity.
        \item Equations of motion provide mathematical models of motion.
        \item Understanding these concepts builds the foundation for more complex topics.
    \end{itemize}
\end{frame}

\begin{frame}
    \centering
    \textbf{Thank you!}\\
    Questions?
\end{frame}

\end{document}
