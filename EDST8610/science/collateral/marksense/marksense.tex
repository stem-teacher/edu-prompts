\documentclass[11pt, a4paper]{article}
\usepackage{geometry}
\geometry{left=2cm, right=2cm, top=2cm, bottom=2cm}
\usepackage{hyperref} % For potential future links
\usepackage{array}    % Used by template, kept for consistency
\usepackage[utf8]{inputenc} % Handle UTF-8 input
\usepackage[T1]{fontenc} % Modern font encoding

% --- Document Specific Information ---
\title{MarkSense - Teacher's Guide to Automated Assessment}
% \date{\today} % Optional: Uncomment for today's date, or remove for no date
\date{} % Set empty date as per template style likely intended
\author{} % Set empty author as per template style likely intended (or add author if needed)

\begin{document}
\maketitle
\vspace{-2em} % Reduce space after title, as per template

% --- Introduction ---
\section*{Introduction}

MarkSense streamlines the assessment process, providing near-instantaneous, detailed feedback to your students while significantly reducing your manual marking time. This guide outlines the simple workflow from creating your assessment to delivering graded results.

% --- Workflow Overview ---
\section*{Workflow Overview}
\begin{enumerate}
    \item \textbf{Prepare Assessment:} Create your assessment document and easily generate the corresponding MarkSense template using the built-in tools. Establish the marking rubric for AI assistance.
    \item \textbf{Distribute:} Give the assessment handout (PDF or print) to students.
    \item \textbf{Scan:} Students scan their completed handouts using the classroom scanner linked to MarkSense.
    \item \textbf{Automated Marking:} MarkSense automatically processes each scan: identifies the student and assessment, marks multiple-choice questions, uses AI to analyze written answers against your rubric, generates feedback comments, calculates scores, and compiles a personalized, annotated feedback document for the student.
    \item \textbf{Deliver Feedback:} The graded feedback document is automatically emailed to each student.
    \item \textbf{Review (Optional):} Use the MarkSense application to quickly review any flagged items (\texttt{e.g.}, answers the AI was unsure about) or browse overall class results.
\end{enumerate}

% --- Step-by-Step Guide ---
\section*{Step-by-Step Guide for Teachers}
\begin{enumerate} % Main steps 1-4
    \item \textbf{Creating Your Assessment:}
        \begin{itemize} % Sub-points within step 1
            \item \textbf{Design Handout:} Prepare your assessment document as usual (\texttt{e.g.}, in \texttt{Word}, \texttt{Google Docs}, or using Markdown with \LaTeX{} for scientific content). Ensure clear areas for student responses (MC bubbles, written answer spaces). Export the final handout as a PDF (\texttt{e.g.}, \texttt{Year7\_Science\_Term1\_Exam.pdf}).
            \item \textbf{Use MarkSense Setup:} Open the MarkSense application. Select "Create New Assessment Template".
                \begin{itemize} % Sub-sub-points for MarkSense setup actions
                    \item Upload your assessment handout PDF (\texttt{Year7\_Science\_Term1\_Exam.pdf}).
                    \item MarkSense will analyze the document. Use the simple on-screen tools to:
                        \begin{itemize} % Further nesting for tool actions
                            \item Confirm or adjust the areas identified for multiple-choice questions and written answers.
                            \item Specify the correct answer for each multiple-choice question (\texttt{e.g.}, click the 'C' bubble for Question 1). MarkSense automatically creates the necessary green/red marking guides in the background template.
                            \item Define where scores and overall/section comments should appear on the feedback document.
                            \item Link or input your marking rubric for each written question to guide the AI analysis.
                        \end{itemize} % End nesting for tool actions
                    \item Save the template within MarkSense. It will automatically associate this template with the assessment handout.
                \end{itemize} % End sub-sub-points
            \item \textbf{Distribute Handout:} Provide the clean assessment handout PDF (or printouts) to your students. The handout includes an embedded identifier (like a subtle QR code) automatically added by MarkSense during setup, linking it to the template and rubric.
        \end{itemize} % End sub-points for step 1

    \item \textbf{Student Completion and Scanning:}
        \begin{itemize} % Sub-points within step 2
            \item Students complete the handout.
            \item Students scan their completed handout using the designated MarkSense classroom scanner. The system automatically reads the embedded identifier to know which student and assessment it is, and which template/rubric to use. No manual linking is needed.
        \end{itemize} % End sub-points for step 2

    \item \textbf{Automated Marking and Feedback:}
        \begin{itemize} % Sub-points within step 3
            \item MarkSense processes each scan immediately.
            \item It marks the MC section, identifying correct/incorrect answers visually on the PDF.
            \item It analyzes written answers using the AI and your provided rubric, assigning scores and generating initial feedback comments based on the rubric criteria.
            \item It calculates section and total scores.
            \item It compiles the annotated PDF with all marks, scores, and comments.
            \item It automatically emails the completed feedback PDF to the student (using email addresses linked to student IDs in your class lists).
        \end{itemize} % End sub-points for step 3

    \item \textbf{Reviewing Results (Optional):}
        \begin{itemize} % Sub-points within step 4
            \item Open the MarkSense application at any time.
            \item Select the class and assessment you want to review.
            \item Browse through the digitally marked student papers.
            \item MarkSense may flag specific answers (\texttt{e.g.}, ambiguous MC marks, low AI confidence on a written answer) for your quick review. You can easily override AI scores or add your own clarifying comments directly in the app.
            \item View class analytics and generate reports (\texttt{e.g.}, \texttt{CSV} scores for upload).
        \end{itemize} % End sub-points for step 4
\end{enumerate} % End main steps 1-4

% --- Benefits Section ---
\section*{Benefits for You}
\begin{itemize}
    \item \textbf{Save Time:} Dramatically reduces time spent on repetitive marking tasks.
    \item \textbf{Focus on Teaching:} Spend more time analyzing results, planning interventions, and interacting with students.
    \item \textbf{Consistent Marking:} Ensures all papers are marked against the same standard using your defined rubric.
    \item \textbf{Faster Feedback:} Students receive detailed feedback almost immediately, reinforcing learning.
\end{itemize}

% --- Closing ---
MarkSense handles the technical details, allowing you to focus on creating quality assessments and providing timely, meaningful feedback.

\end{document}
