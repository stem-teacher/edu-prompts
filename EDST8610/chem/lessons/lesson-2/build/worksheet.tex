\documentclass[11pt, a4paper]{article}
\usepackage{geometry}
\geometry{left=2cm, right=2cm, top=2cm, bottom=2cm}
\usepackage{amsmath}
\usepackage{graphicx}
\usepackage{array}
\usepackage[utf8]{inputenc}
\usepackage[T1]{fontenc}
\usepackage{hyperref}
\usepackage{mhchem}

\title{Year 12 Chemistry - Worksheet 2 \\ Navigating Pathways: Short Syntheses}
\date{Module 7 - Lesson 2}
\author{Student Name: \underline{\hspace{5cm}} ID: \underline{\hspace{3cm}}}

\begin{document}
\maketitle

\section*{Part 1: New Reactions - Oxidation \& Esterification}

1.  Oxidation of Alcohols (Node: CHM\_M7\_ALC\_N6):
    \begin{itemize}
        \item[a)] What type of alcohol (primary, secondary, tertiary) is oxidised to form an aldehyde (which can be further oxidised to a carboxylic acid)? \vspace{0.5cm}
        \item[b)] What type of alcohol is oxidised to form a ketone? \vspace{0.5cm}
        \item[c)] What type of alcohol is generally resistant to oxidation under these conditions? \vspace{0.5cm}
        \item[d)] Name a common oxidising agent used for these reactions (often represented as [O]). \vspace{0.5cm}
    \end{itemize}

2.  Esterification (Node: CHM\_M7\_ESTER\_N1):
    \begin{itemize}
        \item[a)] What two types of functional groups react to form an ester? \vspace{0.5cm}
        \item[b)] What catalyst and condition are typically required for esterification? \vspace{0.5cm}
        \item[c)] What small molecule is also produced during esterification? \vspace{0.5cm}
    \end{itemize}

3.  Ester Naming (Node: CHM\_M7\_NOM\_N11): Name the following esters:
    \begin{itemize}
        \item[a)] The ester formed from methanol and ethanoic acid: \underline{\hspace{5cm}}
        \item[b)] The ester formed from propan-1-ol and propanoic acid: \underline{\hspace{5cm}}
        \item[c)] Draw the structure of ethyl propanoate: \vspace{2cm}
    \end{itemize}

\section*{Part 2: Planning Short Syntheses}

\textbf{Instructions:} For the synthesis problems assigned in class (see Activity Sheet 2), use the space below to plan your pathway. Use the Chord Diagram tool to help identify intermediates and reaction types. For each step, show: Reactant Structure/Name $\xrightarrow{\text{Reagent(s)/Conditions}}$ Product Structure/Name.

\subsection*{Problem 1: [e.g., Ethene to Ethanoic Acid]}
\textbf{Planning Notes (Intermediates? Reaction Types?):}
\vspace{1cm}

\textbf{Pathway:}
Step 1:
\vspace{2cm}
Step 2:
\vspace{2cm}

\subsection*{Problem 2: [e.g., Propane to Propanone]}
\textbf{Planning Notes:}
\vspace{1cm}

\textbf{Pathway:}
Step 1:
\vspace{2cm}
Step 2:
\vspace{2cm}
(Add more steps if needed)

\subsection*{Problem 3: [e.g., 1-Bromobutane to Butanoic Acid]}
\textbf{Planning Notes:}
\vspace{1cm}

\textbf{Pathway:}
Step 1:
\vspace{2cm}
Step 2:
\vspace{2cm}
(Add more steps if needed)

\end{document}
