\documentclass[11pt, a4paper]{article}
\usepackage{geometry}
\geometry{left=2cm, right=2cm, top=2cm, bottom=2cm}
\usepackage{hyperref}
\usepackage{array}
\usepackage[utf8]{inputenc}
\usepackage[T1]{fontenc}
\usepackage{graphicx} % Added for potential images

\title{Year 12 Chemistry - Lesson Plan 2/3 \\ Navigating Pathways: Connecting Reactions for Short Syntheses}
\date{Module 7: Organic Chemistry (Approx. Week 6/7)}
\author{Mr Haynes} % Replace with your name if needed

\begin{document}
\maketitle
\vspace{-2em} % Reduce space after title

\section*{Lesson Overview}
\begin{itemize}
    \item \textbf{Lesson Title:} Navigating Pathways: Connecting Reactions for Short Syntheses
    \item \textbf{Duration:} 60 minutes
    \item \textbf{Focus Inquiry Question:} How are different classes of organic compounds interconverted through reaction pathways? (Focus on 2-3 step sequences)
    \item \textbf{Placement:} Assumes L1 completed. Assumes students have now learned Alcohol Oxidation (CHM\_M7\_ALC\_N6) and Esterification (CHM\_M7\_ESTER\_N1, CHM\_M7\_NOM\_N11).
\end{itemize}

\section*{Syllabus Alignment \& Knowledge Nodes Targeted}
\begin{itemize}
    \item \textbf{Outcomes:} CH12-14 (Predicts reactions involving carbon compounds), CH11/12-6 (Solves scientific problems - synthesis planning), CH11/12-7 (Communicates understanding).
    \item \textbf{Content:} Applying knowledge of multiple reaction steps in sequence. Introduction to synthesis planning logic.
    \item \textbf{Knowledge Nodes (Focus):} CHM\_M7\_ALC\_N6 (Oxidation), CHM\_M7\_ESTER\_N1 (Esterification), CHM\_M7\_NOM\_N11 (Ester Naming). Review/Application of nodes from L1.
\end{itemize}

\section*{Student Learning Objectives (Aligned with Nodes \& Cognitive Strategies)}
Students will be able to:
\begin{itemize}
    \item Identify multi-step pathways (2-3 steps) between functional groups using the chord-diagram tool. [Apply - S2 Visualisation]
    \item Propose a logical sequence of known reactions to achieve a simple synthetic transformation. [Analyse - CH11/12-6]
    \item Identify necessary reagents and conditions for each step in a proposed short synthesis. [Apply - CH12-14]
    \item Name reactants, intermediates, and products (including simple esters) in a short synthesis pathway. [Apply - CHM\_M7\_NOM\_N11]
    \item Articulate the planning process for solving a simple synthesis problem. [Apply - S6 Metacognition]
    \item Update and utilise their reaction map (mental or visual) incorporating oxidation and esterification. [Apply - S3 Concept Mapping]
    \item [Literacy] Describe a short synthesis pathway using correct terminology and reaction representation.
\end{itemize}

\section*{Lesson Structure \& Activities}

\subsection*{Introduction \& Retrieval Practice (10 mins)}
\begin{itemize}
    \item \textbf{Teacher Activity:} Brief retrieval quiz: "Show the reagents needed to convert 1-chloropropane to propan-1-ol." "What functional group results from dehydrating an alcohol?". Briefly review the chord diagram from L1, highlighting the previously learned connections. Add new nodes/reactions (Oxidation, Esterification) to the discussion/visualisation. [S3 Update Map, supports S4]
    \item \textbf{Student Activity:} Answer retrieval questions. Observe map update.
    \item \textbf{Pedagogy Focus:} Retrieval Practice, Activation of Prior Knowledge, Integrating New Knowledge into Schema (S3).
\end{itemize}

\subsection*{Modelling Short Synthesis Planning (15 mins)}
\begin{itemize}
    \item \textbf{Teacher Activity:} Explicitly model solving a 2-step synthesis problem (e.g., Ethene $\rightarrow$ Ethanoic Acid).
        \begin{itemize}
            \item Step 1: Identify Start/End functional groups.
            \item Step 2: Use the chord diagram tool to find path(s) - "Ethene to Alcohol (Addition), then Alcohol to Carboxylic Acid (Oxidation)". Identify intermediate (Ethanol).
            \item Step 3: Write out the sequence, adding specific reagents/conditions for each step learned previously (hover on tool or recall).
            \item Step 4: "Think aloud" the metacognitive process: "I need to get from alkene to acid. The map shows I can go via alcohol. First step is hydration... second step is oxidation of a primary alcohol..." [S1 Explicit Instruction, S2 Visualisation, S6 Metacognition]
        \end{itemize}
    \item \textbf{Student Activity:} Follow the modelled example. Ask clarifying questions.
    \item \textbf{Pedagogy Focus:} Modelling Problem-Solving Process, Explicit use of Visualisation Tool for Planning, Metacognitive Scaffolding (S6).
\end{itemize}

\subsection*{Paired Problem Solving (25 mins)}
\begin{itemize}
    \item \textbf{Teacher Activity:} Provide pairs of students with simple 2 or 3-step synthesis problems (via Activity Sheet 2 or whiteboard). Encourage them to use the chord diagram tool *first* to plan their route, then write down the detailed steps (structures/names, reagents/conditions) on mini-whiteboards or paper. Circulate, prompt with planning questions ("What intermediate is needed?", "What reaction achieves that?", "Check the tool for reagents."). [S4 Interleaving if problems mix reaction types]
    \item \textbf{Student Activity:} Work in pairs. Use the visualisation tool to plan pathways for given synthesis problems. Draft the reaction sequence with structures, names, reagents, conditions. Discuss strategy with partner.
    \item \textbf{Pedagogy Focus:} Collaborative Learning, Applying Knowledge, Problem Solving (CH11/12-6), Active use of Visualisation Tool, Metacognitive Practice (S6 prompts).
    \item \textbf{ICT Integration:} Chord Diagram Tool (Student Use), Devices.
\end{itemize}

\subsection*{Sharing & Consolidation (10 mins)}
\begin{itemize}
    \item \textbf{Teacher Activity:} Select 1-2 pairs to present their pathway for one problem on the board. Facilitate discussion – "Did anyone find a different route using the tool?". Briefly summarise the process of using the map for planning.
    \item \textbf{Student Activity:} Share proposed pathway if selected. Observe and comment on others' pathways. Ask questions.
    \item \textbf{Pedagogy Focus:} Communicating Solutions (CH11/12-7), Peer Learning, Consolidating the Planning Strategy.
\end{itemize}

\section*{Resources Required}
\begin{itemize}
    \item Interactive Chord Diagram Visualisation Tool.
    \item Projector / Whiteboard.
    \item Student devices with internet access.
    \item Mini-whiteboards or paper for drafting pathways.
    \item Activity Sheet 2 (containing 2/3-step synthesis problems - see below).
\end{itemize}

\section*{Assessment}
\begin{itemize}
    \item \textbf{Formative:} Observation of paired problem-solving process (use of tool, discussion, logic). Review of drafted pathways on mini-whiteboards/paper (checking for correct intermediates, reagents, conditions). Quality of participation in sharing session.
\end{itemize}

\section*{Differentiation}
\begin{itemize}
    \item \textbf{Support:} Provide partially completed pathways (e.g., give the intermediate). Offer a list of possible reagents to choose from. Pair students strategically.
    \item \textbf{Extension:} Challenge students to find the *shortest* possible route if multiple exist. Ask them to propose a synthesis for a slightly more complex target requiring 3 steps.
\end{itemize}

\end{document}
