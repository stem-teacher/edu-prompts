\documentclass[11pt, a4paper]{article}
\usepackage{geometry}
\geometry{left=2cm, right=2cm, top=2cm, bottom=2cm}
\usepackage{hyperref}
\usepackage{array}
\usepackage[utf8]{inputenc}
\usepackage[T1]{fontenc}
\usepackage{graphicx} % Added for potential images

\title{Year 12 Chemistry - Lesson Plan 3/3 \\ Designing \& Communicating Syntheses: Mastering Flowcharts}
\date{Module 7: Organic Chemistry (Approx. Week 9/10)}
\author{Mr Haynes} % Replace with your name if needed

\begin{document}
\maketitle
\vspace{-2em} % Reduce space after title

\section*{Lesson Overview}
\begin{itemize}
    \item \textbf{Lesson Title:} Designing \& Communicating Syntheses: Mastering Flowcharts
    \item \textbf{Duration:} 60 minutes
    \item \textbf{Focus Inquiry Question:} How can we represent multi-step organic syntheses? (Culminating task)
    \item \textbf{Placement:} Assumes L1 \& L2 completed. Assumes all relevant reactions from EduKG subset (Alkene Add/Sub, Haloalkane Sub, Alcohol Sub/Dehyd/Oxid, Esterification) have been covered in the module.
\end{itemize}

\section*{Syllabus Alignment \& Knowledge Nodes Targeted}
\begin{itemize}
    \item \textbf{Outcomes:} CH12-14 (Analyses structure, predicts reactions), CH11/12-6 (Solves scientific problems - complex synthesis), CH11/12-7 (Communicates scientific understanding - flowchart conventions).
    \item \textbf{Content:} Constructing multi-step synthesis flowcharts with correct conventions, reagents, and conditions. Applying integrated knowledge of reaction pathways.
    \item \textbf{Knowledge Nodes (Focus):} CHM\_M7\_SYNTH\_N1 (Draft and construct flow charts...). Application of all prerequisite reaction nodes.
\end{itemize}

\section*{Student Learning Objectives (Aligned with Nodes \& Cognitive Strategies)}
Students will be able to:
\begin{itemize}
    \item Analyse a multi-step synthesis problem to identify starting material, target product, and required functional group changes. [Analyse - CH11/12-6]
    \item Utilise the chord-diagram tool and reaction knowledge to devise a logical multi-step synthesis pathway. [Create - CH11/12-6]
    \item Apply standard flowchart conventions to represent a multi-step synthesis accurately. [Apply - CHM\_M7\_SYNTH\_N1, CH11/12-7]
    \item Include correct structural formulae (or IUPAC names) for all intermediates and products in a flowchart. [Apply - CH11/12-7]
    \item Specify correct reagents and conditions for each step in the flowchart. [Apply - CH12-14]
    \item Justify the chosen reaction pathway and steps. [Evaluate - S6 Metacognition]
    \item [Literacy] Communicate a complex chemical process clearly and conventionally using a flowchart.
    \item [Numeracy] Implicitly check atom conservation through balanced steps/structures.
\end{itemize}

\section*{Lesson Structure \& Activities}

\subsection*{Introduction \& Map Review (10 mins)}
\begin{itemize}
    \item \textbf{Teacher Activity:} Briefly review the "complete" reaction map using the chord diagram, emphasizing the network of possibilities learned over the module. Pose a quick challenge question: "Can we directly convert an alkane to an ester using the reactions we've learned? Why/why not? What intermediate(s) would be needed?". State lesson objective: To plan complex syntheses and communicate them using formal flowcharts. [S3 Map Review]
    \item \textbf{Student Activity:} Participate in map review and challenge question discussion.
    \item \textbf{Pedagogy Focus:} Consolidation of Schema (S3), Assessing Integrated Understanding.
\end{itemize}

\subsection*{Instruction \& Modelling: Flowchart Construction (15 mins)}
\begin{itemize}
    \item \textbf{Teacher Activity:} Explicitly teach standard flowchart conventions for chemical synthesis (referencing CHM\_M7\_SYNTH\_N1 literacy skills): Boxes for compounds (structure or name), Arrows for reactions, Reagents/conditions written above/below arrows. Model solving a more complex (e.g., 3-step) synthesis problem (e.g., Propane $\rightarrow$ Propan-1-ol $\rightarrow$ Propanal $\rightarrow$ Propanoic Acid).
        \begin{itemize}
            \item First, use the chord diagram tool to plan the route, "thinking aloud" the choices.
            \item Then, translate the planned route step-by-step into a correctly formatted flowchart on the board/projector. Emphasise clarity and inclusion of all required details. [S1 Explicit Instruction, S2 Visualisation, S6 Metacognition]
        \end{itemize}
    \item \textbf{Student Activity:} Take notes on flowchart conventions (can be added to Worksheet 3). Follow the modelled example. Ask questions about conventions or the synthesis logic.
    \item \textbf{Pedagogy Focus:} Explicit Teaching of Communication Conventions (CH11/12-7), Modelling the Link between Planning (Tool) and Communicating (Flowchart).
\end{itemize}

\subsection*{Group Synthesis Challenge (25 mins)}
\begin{itemize}
    \item \textbf{Teacher Activity:} Divide students into small groups (3-4). Provide each group with a different, challenging synthesis problem (e.g., starting from a simple alkane/alkene, synthesise a specific ester or ketone) via Activity Sheet 3. Provide access to the chord diagram tool for planning and materials for flowchart construction (large paper, mini-whiteboards, or digital tool). Instruct groups to produce a complete, conventional flowchart as their solution. Circulate to facilitate, guide, and ask probing questions (S6 prompts). [Problem-Based Learning element, S4 Interleaving via varied problems]
    \item \textbf{Student Activity:} Work collaboratively in groups. Analyse the assigned problem. Use the chord diagram tool to brainstorm and plan a viable pathway. Construct the synthesis flowchart, ensuring correct structures/names, reagents, conditions, and conventions. Discuss and justify steps within the group.
    \item \textbf{Pedagogy Focus:} Collaborative Problem Solving (CH11/12-6), Application of Integrated Knowledge (Create Level), Practice with Communication Conventions (CH11/12-7), Active Learning.
    \item \textbf{ICT Integration:} Chord Diagram Tool (Student Use), Optional Digital Whiteboard/Drawing Tool.
\end{itemize}

\subsection*{Peer Review / Gallery Walk (10 mins)}
\begin{itemize}
    \item \textbf{Teacher Activity:} Facilitate a gallery walk or structured peer review. Groups display their flowcharts. Provide a simple peer review checklist (on Activity Sheet 3 or separate) focusing on: Logical sequence, Correct structures/names, Correct reagents/conditions, Clear conventions. Encourage constructive feedback. Briefly summarise common successes/errors observed. Assign Exit Ticket.
    \item \textbf{Student Activity:} Display group flowchart. View and provide feedback on other groups' flowcharts using the checklist. Discuss feedback received. Complete Exit Ticket.
    \item \textbf{Pedagogy Focus:} Peer Assessment, Reinforcing Communication Standards (CH11/12-7), Identifying Common Errors, Formative Assessment.
\end{itemize}

\section*{Resources Required}
\begin{itemize}
    \item Interactive Chord Diagram Visualisation Tool.
    \item Projector / Whiteboard.
    \item Student devices (for tool access).
    \item Materials for flowchart construction (large paper/posters, markers, mini-whiteboards, or digital collaborative whiteboard).
    \item Activity Sheet 3 (containing synthesis challenge problems and peer review checklist - see below).
    \item Worksheet 3 (summary of flowchart conventions - see below).
    \item Exit Ticket question (prepared separately, e.g., "Outline the first two steps (reactants, reagents) needed to convert but-1-ene into ethyl ethanoate.").
\end{itemize}

\section*{Assessment}
\begin{itemize}
    \item \textbf{Formative:} Observation of group collaboration, planning strategies, and flowchart construction. Evaluation of final group flowcharts based on logic, accuracy, detail, and conventions. Quality of peer feedback provided. Analysis of Exit Ticket responses.
\end{itemize}

\section*{Differentiation}
\begin{itemize}
    \item \textbf{Support:} Assign specific roles within groups (e.g., planner, recorder, reagent checker). Provide partially solved synthesis problems or flowcharts with gaps to fill. Offer a checklist of key reactions.
    \item \textbf{Extension:} Challenge groups to find the most efficient (fewest steps) pathway. Ask them to propose a synthesis for a target molecule not directly derivable using only the learned reactions, requiring slight modification or an additional step. Ask groups to include potential side reactions or yield considerations (conceptual).
\end{itemize}

\end{document}
