\documentclass[11pt, a4paper]{article}
\usepackage{geometry}
\geometry{left=2cm, right=2cm, top=2cm, bottom=2cm}
\usepackage{graphicx}
\usepackage{hyperref}
\usepackage{array}
\usepackage[utf8]{inputenc}
\usepackage[T1]{fontenc}
\usepackage{mhchem}

\title{Year 12 Chemistry - Activity Sheet 3 \\ Group Synthesis Challenge \& Flowchart Peer Review}
\date{Module 7 - Lesson 3}
\author{Organic Chemistry}

\begin{document}
\maketitle

\section*{Aim}
To apply your integrated knowledge of organic reactions to design a multi-step synthesis pathway and communicate it using a conventional flowchart. To provide constructive feedback on peers' flowcharts.

\section*{Part A: Group Synthesis Challenge}

\textbf{Instructions:}
Work in your assigned group. Your task is to design a synthesis pathway for the problem below and represent it as a detailed, accurate flowchart.
\begin{enumerate}
    \item Analyse the starting material and target product.
    \item Use the Chord Diagram Visualisation tool and your knowledge to plan a logical sequence of reactions.
    \item Construct a flowchart on the provided materials (paper/whiteboard/digital).
    \item Ensure your flowchart includes:
        \begin{itemize}
            \item Correct structures or IUPAC names for all compounds (starting material, intermediates, product) in boxes.
            \item Arrows clearly indicating each reaction step.
            \item Accurate reagents and conditions listed for every step on the arrows.
            \item Adherence to standard flowchart conventions.
        \end{itemize}
    \item Be prepared to justify your chosen pathway.
\end{enumerate}

\subsection*{Challenge Problem (Teacher will assign one per group)}

\textbf{Problem 1:} Design a synthesis pathway to produce **ethyl propanoate** starting from **ethene** and **propane**. (Assume necessary inorganic reagents are available).

\vspace{0.5cm} \hrulefill \vspace{0.5cm}

\textbf{Problem 2:} Design a synthesis pathway to produce **butanone** starting from **but-1-ene**.

\vspace{0.5cm} \hrulefill \vspace{0.5cm}

\textbf{Problem 3:} Design a synthesis pathway to produce **1,2-dichloroethane** starting from **ethanol**.

\vspace{0.5cm} \hrulefill \vspace{0.5cm}

\textbf{Problem 4:} Design a synthesis pathway to produce **propanoic acid** starting from **propane**.

\section*{Part B: Flowchart Peer Review Checklist}

\textbf{Instructions:} When reviewing another group's flowchart, consider the following criteria:

\begin{tabular}{|l|p{8cm}|c|}
\hline
\textbf{Criteria} & \textbf{Description} & \textbf{Check ($\checkmark$)} \\ \hline
\textbf{1. Logical Pathway} & Does the sequence of reactions make chemical sense to get from start to target? & \\ \hline
\textbf{2. Correct Structures/Names} & Are the structures or names shown for reactants, intermediates, and products accurate? & \\ \hline
\textbf{3. Correct Reagents/Conditions} & Are the specified reagents and conditions appropriate for each reaction step shown? & \\ \hline
\textbf{4. Flowchart Conventions} & Are compounds in boxes? Are arrows used correctly? Are reagents/conditions placed appropriately on arrows? & \\ \hline
\textbf{5. Clarity \& Neatness} & Is the flowchart easy to read and understand? & \\ \hline
\end{tabular}

\vspace{1cm}
\textbf{Constructive Feedback / Comments:}
\vspace{2cm}

\end{document}
