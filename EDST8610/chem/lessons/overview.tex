## 2. Lesson Sequence Overview and Rationale (Narrative)


This three-lesson sequence is designed to be integrated within the NSW HSC Chemistry Module 7: Organic Chemistry, specifically targeting the understanding of functional group interconversions and the culminating skill of designing and representing multi-step syntheses (CHM\_M7\_SYNTH\_N1).

The core challenge in this topic, identified in educational research, is the "schema gap": students often learn individual reactions in isolation but struggle to connect them into the coherent network required for synthesis planning. This leads to cognitive overload and difficulty transferring knowledge. A purely linear teaching approach often exacerbates this issue.

Therefore, this sequence adopts a strategy grounded in cognitive science principles (Cognitive Load Theory, Schema Theory, Dual Coding) and leverages a dynamic chord-diagram visualisation tool. The rationale is to:

1.  **Build Schema Explicitly:** Use the visualisation tool as a central "reaction map" from early on (Lesson 1, approx. Week 4/5), making the interconnectedness of functional groups and reactions visually explicit (Strategy 3: Concept Mapping).
2.  **Manage Cognitive Load:** The visualisation helps organise the complex network of reactions, reducing the cognitive load associated with recalling disparate facts. Information (reagents/conditions) is revealed progressively (e.g., on hover), preventing overload.
3.  **Scaffold Complexity:** The sequence builds complexity gradually:
    *   **Lesson 1:** Focuses on understanding the map structure and tool use with familiar, recently learned reactions.
    *   **Lesson 2:** Progresses to planning short (2-3 step) pathways, using the tool for navigation and applying knowledge in simple sequences.
    *   **Lesson 3:** Addresses the syllabus requirement for complex flowchart construction, using the visualisation tool for strategic planning and the flowchart for formal communication.
4.  **Integrate Cognitive Strategies:** The lessons incorporate Explicit Instruction/Worked Examples (S1), Dual Coding/Visualisation (S2), Concept Mapping/Flowcharting (S3), Guided Inquiry/Predictive Exercises (S5), and Metacognitive Prompts (S6). Spaced Practice/Interleaving (S4) is achieved by embedding these lessons within the broader module timeline.

This approach aims to move beyond rote memorisation towards developing a flexible, interconnected understanding, enabling students to effectively solve multi-step synthesis problems and communicate their solutions clearly, aligning with syllabus outcomes CH12-14, CH11/12-6, and CH11/12-7.
