\documentclass[11pt, a4paper]{article}
\usepackage{geometry}
\geometry{left=2cm, right=2cm, top=2cm, bottom=2cm}
\usepackage{amsmath} % For math
\usepackage{graphicx} % For images if needed
\usepackage{array}
\usepackage[utf8]{inputenc}
\usepackage[T1]{fontenc}
\usepackage{hyperref}
\usepackage{mhchem} % Added for chemistry

\title{Year 12 Chemistry - Worksheet 1 \\ Mapping the Territory: Visualising Reactions}
\date{Module 7 - Lesson 1}
\author{Student Name: \underline{\hspace{5cm}} ID: \underline{\hspace{3cm}}} % Placeholder for student info

\begin{document}
\maketitle

\section*{Part 1: The Reaction Map Concept}

1.  In your own words, what is a "reaction map" in organic chemistry? Why might it be useful?
    \vspace{2cm}

2.  The chord diagram tool uses nodes and chords/links. What does each represent?
    \begin{itemize}
        \item Nodes (Outer segments): \underline{\hspace{6cm}}
        \item Chords / Links (Inner connecting bands): \underline{\hspace{6cm}}
    \end{itemize}

\section*{Part 2: Exploring Known Connections with the Tool}

\textbf{Instructions:} Use the interactive Chord Diagram Visualisation tool provided by your teacher to answer the following questions. Focus on the reactions you have already learned (connecting Alkanes, Alkenes, Haloalkanes, Alcohols).

3.  Find the node for \textbf{Alkenes}.
    \begin{itemize}
        \item[a)] List the functional groups that Alkenes are shown to be directly connected to via reactions \textit{you have learned so far}.
        \vspace{1cm}
        \item[b)] Select one of these connections (e.g., Alkene $\rightarrow$ Alcohol). What is the reaction type indicated by the chord colour/filter?
        \vspace{0.5cm}
        \item[c)] Hover over this specific chord/link. What reagent(s) and/or conditions are displayed for this transformation (e.g., \ce{H2O / H+})?
        \vspace{1cm}
    \end{itemize}

4.  Find the node for \textbf{Alcohols}.
    \begin{itemize}
        \item[a)] According to the map (based on reactions learned), can an Alcohol be directly converted back to an Alkene? (Yes/No)
        \item[b)] If yes, what is the reaction type and the key reagent/condition shown when you hover over that link?
        \vspace{1cm}
        \item[c)] According to the map (based on reactions learned), can an Alcohol be directly converted to a Haloalkane? (Yes/No)
        \item[d)] If yes, what reagent is shown for this transformation?
        \vspace{1cm}
    \end{itemize}

5.  Find the connection between \textbf{Haloalkanes} and \textbf{Alcohols}.
    \begin{itemize}
        \item[a)] What reagent is needed to convert a Haloalkane to an Alcohol according to the tool?
        \vspace{0.5cm}
        \item[b)] What type of reaction is this (use colour/filter)?
        \vspace{0.5cm}
    \end{itemize}

6.  Based on your exploration, draw a simple diagram below showing only the nodes for Alkane, Alkene, Haloalkane, and Alcohol, and draw arrows representing the direct, one-step reactions you have identified between them using the tool. Label each arrow with the reaction type (e.g., Addition, Substitution, Dehydration). [S3 Concept Mapping]
    \vspace{5cm}


\end{document}
