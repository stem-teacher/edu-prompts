\documentclass[11pt, a4paper]{article} % Using article class for simpler activity sheet
\usepackage{geometry}
\geometry{left=2cm, right=2cm, top=2cm, bottom=2cm}
\usepackage{graphicx}
\usepackage{hyperref}
\usepackage{array}
\usepackage[utf8]{inputenc}
\usepackage[T1]{fontenc}
\usepackage{mhchem} % Added for chemistry

\title{Year 12 Chemistry - Activity Sheet 1 \\ Guided Exploration: Reaction Visualisation Tool}
\date{Module 7 - Lesson 1}
\author{Organic Chemistry}

\begin{document}
\maketitle

\section*{Aim}
To familiarise yourself with the interactive Chord Diagram Visualisation tool and use it to identify known organic reaction pathways and their key details.

\section*{Tool Access}
Use the link or application provided by your teacher to access the Chord Diagram tool on your device.

\section*{Part A: Tool Familiarisation}
\begin{itemize}
    \item Identify the main components: Outer segments (Nodes), Inner bands (Chords/Links).
    \item Locate the legend or checkboxes that explain the colour-coding for reaction types (e.g., Addition, Substitution, Elimination, Oxidation, Condensation).
    \item Practice hovering your mouse cursor over different Nodes and Chords. Observe what information appears (e.g., Functional group name, Reaction details like reagents/conditions).
    \item Practice using the checkboxes (if available) to filter the view and show only certain reaction types.
\end{itemize}

\section*{Part B: Guided Exploration (Complete relevant sections on Worksheet 1)}

\textbf{Task 1: Focus on Alkenes}
\begin{enumerate}
    \item Click on or interact with the 'Alkene' node.
    \item Observe the chords connecting to it.
    \item For the connection representing the reaction of an Alkene to form an Alcohol:
        \begin{itemize}
            \item Identify the reaction type using the colour code / filter.
            \item Hover to find the specific reagents/conditions (e.g., Hydration conditions). Record these on Worksheet 1.
        \end{itemize}
    \item Repeat for the connection representing the reaction of an Alkene to form a Haloalkane (via HX addition). Record the reaction type and typical reagent (e.g., HBr) on Worksheet 1.
    \item Repeat for the connection representing the reaction of an Alkene to form a Haloalkane (via X2 addition). Record the reaction type and typical reagent (e.g., \ce{Br2}).
\end{enumerate}

\textbf{Task 2: Focus on Alcohols}
\begin{enumerate}
    \item Click on or interact with the 'Alcohol' node.
    \item Find the chord representing the conversion of an Alcohol back to an Alkene.
        \begin{itemize}
            \item Identify the reaction type (Dehydration/Elimination).
            \item Hover to find the specific reagents/conditions. Record these on Worksheet 1.
        \end{itemize}
    \item Find the chord representing the conversion of an Alcohol to a Haloalkane.
        \begin{itemize}
            \item Identify the reaction type (Substitution).
            \item Hover to find the typical reagent. Record this on Worksheet 1.
        \end{itemize}
\end{enumerate}

\textbf{Task 3: Focus on Haloalkanes}
\begin{enumerate}
    \item Click on or interact with the 'Haloalkane' node.
    \item Find the chord representing the conversion of a Haloalkane to an Alcohol.
        \begin{itemize}
            \item Identify the reaction type (Substitution).
            \item Hover to find the typical reagent/conditions. Record these on Worksheet 1.
        \end{itemize}
    \item Does the map (based on reactions learned so far) show a direct conversion from an Alkane to a Haloalkane? Identify the reaction type and condition required.
\end{enumerate}

\textbf{Task 4: Synthesise (Worksheet 1, Q6)}
Use your findings to draw the connections between the four main functional groups studied so far (Alkanes, Alkenes, Haloalkanes, Alcohols) on Worksheet 1.

\end{document}
