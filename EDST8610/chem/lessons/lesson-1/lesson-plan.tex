\documentclass[11pt, a4paper]{article}
\usepackage{geometry}
\geometry{left=2cm, right=2cm, top=2cm, bottom=2cm}
\usepackage{hyperref}
\usepackage{array}
\usepackage[utf8]{inputenc}
\usepackage[T1]{fontenc}
\usepackage{graphicx} % Added for potential images

\title{Year 12 Chemistry - Lesson Plan 1/3 \\ Mapping the Territory: Visualising Functional Group Interconversions}
\date{Module 7: Organic Chemistry (Approx. Week 4/5)}
\author{Mr Haynes} % Replace with your name if needed

\begin{document}
\maketitle
\vspace{-2em} % Reduce space after title

\section*{Lesson Overview}
\begin{itemize}
    \item \textbf{Lesson Title:} Mapping the Territory: Visualising Functional Group Interconversions
    \item \textbf{Duration:} 60 minutes
    \item \textbf{Focus Inquiry Question:} How are different classes of organic compounds interconverted through reaction pathways? (Introduction)
    \item \textbf{Placement:} Assumes students have learned basic nomenclature and reactions connecting Alkanes, Alkenes, Haloalkanes, Alcohols (e.g., Alkene Additions, Alkane/Haloalkane Subs, Alcohol Dehydration/Sub).
\end{itemize}

\section*{Syllabus Alignment \& Knowledge Nodes Targeted}
\begin{itemize}
    \item \textbf{Outcomes:} CH12-14 (Analyses structure, predicts reactions - single steps), CH11/12-7 (Communicates understanding - using representations)
    \item \textbf{Content:} Implicit review of reactions covered to date. Introduction to representing reaction networks.
    \item \textbf{Knowledge Nodes (Focus on Connections):} CHM\_M7\_RPROD\_N1 (Alkene Add), CHM\_M7\_RPROD\_N2 (Alkane Sub), CHM\_M7\_ALC\_N7 (Haloalk->Alc Sub), CHM\_M7\_ALC\_N5 (Alc->Haloalk Sub), CHM\_M7\_ALC\_N4 (Alc Dehyd).
\end{itemize}

\section*{Student Learning Objectives (Aligned with Nodes \& Cognitive Strategies)}
Students will be able to:
\begin{itemize}
    \item Explain the concept of a "reaction map" for organic chemistry. [Understand]
    \item Identify key functional groups (nodes) and known reaction pathways (chords/links) using the chord-diagram visualisation tool. [Apply - S2 Visualisation]
    \item Use the tool's features (hover, filter) to retrieve information about specific known reactions (type, reagents/conditions). [Apply - S2]
    \item Relate the visual representation on the map to symbolic reaction equations learned previously. [Analyse - S2 Dual Coding]
    \item Begin constructing a mental schema of interconnected reactions. [Understand - S3 Concept Mapping Intro]
    \item [Literacy] Use terminology for functional groups and basic reaction types correctly.
\end{itemize}

\section*{Lesson Structure \& Activities}

\subsection*{Introduction (10 mins)}
\begin{itemize}
    \item \textbf{Teacher Activity:} Display Inquiry Question. Quick recap quiz (retrieval practice): "Name the product when HBr adds to propene." "What condition is needed to substitute Cl onto methane?". Introduce the idea that reactions form a network/map, not just a list. State lesson objective: Learning to use a visual tool to explore this map.
    \item \textbf{Student Activity:} Answer recap questions. Listen to introduction.
    \item \textbf{Pedagogy Focus:} Activate Prior Knowledge, Set Context, Retrieval Practice (supports S4).
\end{itemize}

\subsection*{Exploration: Introducing the Visualisation Tool (25 mins)}
\begin{itemize}
    \item \textbf{Teacher Activity:} Introduce and demonstrate the interactive Chord Diagram visualisation tool (projected). Explain: Nodes = Functional Groups, Chords = Reactions. Show how chord colour relates to reaction type (using legend/checkboxes). Demonstrate hover-over feature for reaction details (reagents/conditions). Focus *only* on nodes/reactions students have already learned (e.g., Alkene, Alcohol, Haloalkane connections). Explicitly link tool view to a written equation shown previously. [S1 Explicit Instruction, S2 Visualisation]
    \item \textbf{Student Activity:} Observe demonstration. Ask clarifying questions about the tool.
    \item \textbf{Pedagogy Focus:} Tool Introduction, Linking Visual Representation to Symbolic (Dual Coding S2), Managing Cognitive Load (showing only known parts first).
    \item \textbf{ICT Integration:} Chord Diagram Tool, Projector.
\end{itemize}

\subsection*{Guided Exploration \& Consolidation (25 mins)}
\begin{itemize}
    \item \textbf{Teacher Activity:} Distribute Worksheet 1 and provide link/access to the visualisation tool. Guide students (working individually or pairs on devices) through Worksheet 1 tasks. Circulate, check understanding, and assist with tool usage. Lead brief class discussion on findings (e.g., "What reaction types connect Alcohols and Alkenes according to the map?"). Consolidate the idea of the map as a growing organiser for their knowledge. Assign Exit Ticket. [S3 Concept Mapping Intro, S2 Visualisation]
    \item \textbf{Student Activity:} Use the visualisation tool on devices to complete Worksheet 1 tasks (e.g., identifying connections, retrieving reagent info via hover, filtering). Participate in discussion. Complete Exit Ticket.
    \item \textbf{Pedagogy Focus:} Active Learning, Guided Inquiry using Visualisation Tool, Reinforcing Connections, Formative Assessment.
    \item \textbf{ICT Integration:} Chord Diagram Tool (Student Use), Devices.
\end{itemize}

\section*{Resources Required}
\begin{itemize}
    \item Interactive Chord Diagram Visualisation Tool.
    \item Projector.
    \item Student devices with internet access (recommended 1:1 or 1:2).
    \item Worksheet 1 (separate file - see below).
    \item Exit Ticket questions (prepared separately, e.g., "Using the visualizer, what reagent is shown for converting an Alkene to an Alcohol?").
\end{itemize}

\section*{Assessment}
\begin{itemize}
    \item \textbf{Formative:} Teacher observation of student engagement and tool use. Responses during class discussion. Review of Worksheet 1 answers. Analysis of Exit Ticket responses (checking basic tool interpretation).
\end{itemize}

\section*{Differentiation}
\begin{itemize}
    \item \textbf{Support:} Provide a simplified version of Worksheet 1 with more direct prompts. Pair students for tool exploration. Pre-fill parts of the map diagram on the worksheet.
    \item \textbf{Extension:} Ask students to predict what other connections might exist based on functional group similarities. Challenge students to find a reaction on the map they haven't learned yet and hypothesise about its type.
\end{itemize}

\end{document}
