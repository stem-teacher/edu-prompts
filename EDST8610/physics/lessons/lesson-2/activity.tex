\documentclass[11pt, a4paper]{article} % Using article class for simpler activity sheet
\usepackage{geometry}
\geometry{left=2cm, right=2cm, top=2cm, bottom=2cm}
\usepackage{graphicx}
\usepackage{hyperref}
\usepackage{array}
\usepackage[utf8]{inputenc}
\usepackage[T1]{fontenc}

\title{Year 11 Physics - Activity Sheet 2 \\ Phase Changes \& Heating Curves Simulation}
\date{Module 3 - Lesson 2}
\author{Thermodynamics}

\begin{document}
\maketitle

\section*{Aim}
To use a simulation to observe the relationship between energy input, temperature, and phase changes for water, and to analyse the resulting heating curve quantitatively.

\section*{Knowledge Nodes Targeted}
\begin{itemize}
    \item N1: Temp/KE Relation (Revisited during heating phases)
    \item N3: Specific Heat (Analysing sloped sections of the graph)
    \item N5: Latent Heat (Analysing flat sections of the graph, understanding phase change energy)
\end{itemize}

\section*{ICT Resource: PhET Simulation}

\subsection*{Simulation: States of Matter: Basics}
\begin{itemize}
    \item \textbf{Link:} \href{https://phet.colorado.edu/en/simulations/states-of-matter-basics}{https://phet.colorado.edu/en/simulations/states-of-matter-basics}
    \item \textbf{Setup:}
        \begin{enumerate}
        \item Open the simulation and select the "Phase Changes" screen.
        \item Select "Water" from the top right options.
        \item Observe the initial state (solid ice, likely below 0°C). Note the particle arrangement and motion.
        \item Ensure the thermometer units are set to Celsius (°C).
        \end{enumerate}
\end{itemize}

\subsection*{Procedure \& Data Collection}
\begin{itemize}
    \item \textbf{Heating Process:}
        \begin{enumerate}
        \item Begin adding heat using the slider at the bottom (move towards "Heat"). Try to add heat at a roughly constant rate.
        \item Observe the thermometer reading and the state/motion of the water molecules closely as heat is added.
        \item Continue adding heat until the water has turned into steam and its temperature is significantly above 100°C.
        \end{enumerate}
    \item \textbf{Observations to Focus On:}
        \begin{itemize}
            \item At what temperatures does the phase change from solid to liquid (melting) occur?
            \item At what temperatures does the phase change from liquid to gas (boiling) occur?
            \item What happens to the temperature reading *during* melting?
            \item What happens to the temperature reading *during* boiling?
            \item What happens to the particle motion and arrangement during heating within a single phase (ice, water, or steam)?
            \item What happens to the particle motion and arrangement *during* a phase change?
        \end{itemize}
    \item \textbf{Data Analysis (for Worksheet 2 Part 1):}
        \begin{enumerate}
        \item Sketch the shape of the Temperature vs. Time/Energy graph based on your observations.
        \item Label the different sections corresponding to heating the different phases and the phase changes themselves.
        \item Identify where energy input increases particle kinetic energy (temperature rises) and where it increases potential energy (breaks bonds during phase change).
        \end{enumerate}
\end{itemize}

\section*{Safety Notes}
This is a computer simulation; no physical safety hazards are present.

\end{document}
