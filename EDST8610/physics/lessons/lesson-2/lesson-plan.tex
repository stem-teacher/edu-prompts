\documentclass[11pt, a4paper]{article}
\usepackage{geometry}
\geometry{left=2cm, right=2cm, top=2cm, bottom=2cm}
\usepackage{hyperref}
\usepackage{array}
\usepackage[utf8]{inputenc}
\usepackage[T1]{fontenc}

\title{Year 11 Physics - Lesson Plan 2/3 \\ Thermodynamics: Quantifying Heat & Changing States}
\date{Based on NSW Stage 6 Syllabus (Module 3)}
\author{Philip Haynes}

\begin{document}
\maketitle
\vspace{-2em}

\section*{Lesson Overview}
\begin{itemize}
    \item \textbf{Lesson Title:} Measuring Heat: Specific Heat Capacity and Latent Heat
    \item \textbf{Duration:} 60 minutes
    \item \textbf{Focus Inquiry Question:} How are temperature, thermal energy, and particle motion related? (Specifically how energy input affects temperature vs. state).
\end{itemize}

\section*{Syllabus Alignment & Knowledge Nodes Targeted}
\begin{itemize}
    \item \textbf{Outcomes:} PH11-10, PH11/12-4 (Processing Data), PH11/12-6 (Problem Solving), PH11/12-7 (Communicate)
    \item \textbf{Content:} ACSPH020, Investigate Latent Heat (Syllabus Focus)
    \item \textbf{Knowledge Nodes:} N3 (Specific Heat - Apply), N5 (Latent Heat - Analyse/Apply), links back to N1.
\end{itemize}

\section*{Student Learning Objectives (Aligned with Nodes)}
Students will be able to:
\begin{itemize}
    \item Analyse heat transfer quantitatively using Q=mcΔT (N3 - Apply).
    \item Analyse phase changes quantitatively using Q=mL (N5 - Analyse/Apply).
    \item Interpret heating/cooling curves quantitatively, linking sections to specific heat and latent heat (N5 - Analyse).
    \item Distinguish between the roles of specific heat capacity and latent heat in thermal processes.
    \item [Literacy] Define specific heat capacity (c), latent heat of fusion (Lf), latent heat of vaporization (Lv). Interpret and label phase change graphs accurately (N3, N5).
    \item [Numeracy] Calculate heat energy (Q) using Q=mcΔT and Q=mL. Analyse quantitative data from heating curves (N3, N5).
\end{itemize}

\section*{Lesson Structure \& Activities}

\subsection*{Introduction (10 mins)}
\begin{itemize}
    \item \textbf{Teacher Activity:} Review L1 concepts (Heat/Temp/KE). Pose prompt: "Why does beach sand get hotter than the ocean water under the same sun?" Guide discussion towards the idea that different substances require different amounts of heat for the same temperature change. Introduce Specific Heat Capacity 'c' [N3]. Present Q=mcΔT, focusing on the meaning and role of 'c'.
    \item \textbf{Student Activity:} Participate in recall and discussion. Record definition and concept of specific heat capacity (W/S 2). Understand the variables in Q=mcΔT.
    \item \textbf{Pedagogy Focus:} Retrieval Practice, Linking Concepts, Introducing Quantitative Parameter 'c' (N3).
\end{itemize}

\subsection*{Exploration (30 mins)}
\begin{itemize}
    \item \textbf{Teacher Activity:} Use PhET Simulation "States of Matter: Basics" (See Activity Sheet 2) to demonstrate heating ice through melting and boiling into steam. Project the Temp vs Energy/Time graph. Explicitly guide analysis of the graph: identify rising sections (specific heat) and flat sections (phase change). Ask guiding question: "Energy is being added, but temperature isn't rising. Why?" Introduce Latent Heat 'L' [N5 concept] - energy for changing state/bonds. Define Lf and Lv. Introduce Q=mL formula.
    \item \textbf{Student Activity:} Observe/interact with PhET simulation. Analyse the generated heating curve (guided by W/S 2 Part 1). Discuss the energy's role during phase changes (overcoming intermolecular forces). Record definitions for Lf, Lv, and Q=mL (W/S 2). [N5 Analyse]
    \item \textbf{Pedagogy Focus:} Data Analysis (Graphical N5), Visualisation (Micro/Macro link N1/N5), Guided Inquiry, Cognitive Science: Using simulations to make abstract concepts concrete.
    \item \textbf{ICT Integration:} PhET Simulation.
    \item \textbf{Numeracy Focus:} Interpreting slopes and plateaus on a quantitative graph (N5).
\end{itemize}

\subsection*{Consolidation (20 mins)}
\begin{itemize}
    \item \textbf{Teacher Activity:} Provide clear Worked Examples on board/slide: one using Q=mcΔT [N3 Apply], one using Q=mL [N5 Apply]. Assign practice calculation problems on Worksheet 2 (Part 2). Circulate to provide support. Distribute \#MarkSense Quiz 2.
    \item \textbf{Student Activity:} Follow worked examples. Attempt practice problems on Worksheet 2, applying the correct formula based on the scenario (heating vs. phase change). Complete \#MarkSense Quiz 2 (end of class or homework).
    \item \textbf{Pedagogy Focus:} Cognitive Load Management (Worked Examples), Application Practice (N3, N5 Apply), Formula Selection Skill.
    \item \textbf{Literacy Focus:} Translating word problems into required variables/formulae.
    \item \textbf{Numeracy Focus:} Accurate application of Q=mcΔT and Q=mL (N3, N5).
\end{itemize}

\section*{Resources Required}
\begin{itemize}
    \item Student laptops with internet access.
    \item PhET Simulation link (on Activity Sheet 2).
    \item Worksheet 2 (separate PDF).
    \item \#MarkSense Quiz 2 (included on Worksheet 2 PDF).
    \item Data table with relevant specific heat (c) and latent heat (L) values (e.g., for water/ice/steam, common metals).
    \item Projector/Whiteboard.
\end{itemize}

\section*{Assessment}
\begin{itemize}
    \item \textbf{Formative:} Teacher observation during simulation analysis and problem-solving. Review of Worksheet 2 responses (graph analysis and calculations). Analysis of \#MarkSense Quiz 2 results.
\end{itemize}

\section*{Differentiation}
\begin{itemize}
    \item \textbf{Support:} Provide partially completed worked examples. Allow use of calculators for all steps. Offer formula sheet with clear variable definitions.
    \item \textbf{Extension:} Include multi-step problems involving both specific and latent heat (e.g., ice below 0°C heated to steam above 100°C). Ask students to research Lf/Lv values for other substances and compare.
\end{itemize}

\end{document}
