\documentclass[xcolor=svgnames]{beamer}
\usepackage[utf8]{inputenc}
\usepackage[T1]{fontenc}
\usepackage{xcolor}
\usepackage{booktabs}
\usepackage{amsmath}
\usepackage{graphicx}
\usepackage{siunitx}
\usepackage{hyperref}
\usetheme{Madrid}

% COLORS and FOOTNOTE settings (copy from previous file)
\definecolor{mqred}{RGB}{166, 25, 46}
\definecolor{mqdeepred}{RGB}{118, 35, 47}
\definecolor{mqgray}{RGB}{55, 58, 54}
\definecolor{mqlightgray}{RGB}{237, 235, 229}
\definecolor{mqmagenta}{RGB}{198, 0, 126}
\usecolortheme[named=mqred]{structure}
\setbeamercolor{title in head/foot}{bg=mqlightgray, fg=mqgray}
\setbeamercolor{author in head/foot}{bg=mqdeepred}
\setbeamercolor{page number in head/foot}{bg=mqdeepred, fg=mqlightgray}
\makeatletter
\setbeamertemplate{footline}{
  \leavevmode%
  \hbox{%
  \begin{beamercolorbox}[wd=.5\paperwidth,ht=2.25ex,dp=1ex,center]{author in head/foot}%
    \usebeamerfont{author in head/foot}\insertshortauthor\expandafter\ifblank\expandafter{\beamer@shortinstitute}{}{~~(\insertshortinstitute)}
  \end{beamercolorbox}%
  \begin{beamercolorbox}[wd=.4\paperwidth,ht=2.25ex,dp=1ex,center]{title in head/foot}%
    \usebeamerfont{title in head/foot}\insertshorttitle
  \end{beamercolorbox}%
  \begin{beamercolorbox}[wd=.1\paperwidth,ht=2.25ex,dp=1ex,center]{page number in head/foot}%
    \usebeamerfont{page number in head/foot}\insertframenumber{} / \inserttotalframenumber
  \end{beamercolorbox}}%
  \vskip0pt%
}
\makeatother
\beamertemplatenavigationsymbolsempty


% TITLE, AUTHORS, INSTITUTE, DATE
\title[Thermo: Equilibrium]{Thermodynamics Lesson 3: Equilibrium Calculations and Efficiency}
\author[P. Haynes]{Mr Haynes}
\institute[GHS]{Gosford High School}
\date{\today}

% LOGO (Optional)
%\titlegraphic{\includegraphics[height=2.5cm]{logo.jpg}}

\begin{document}

\begin{frame}
    \titlepage
\end{frame}

\begin{frame}{Outline}
    \tableofcontents
\end{frame}

\section{Review}
\begin{frame}{Review Lesson 2}
    \frametitle{Recap: Quantifying Heat}
    Quick Quiz: Match the scenario to the formula!
    \begin{itemize}
        \item Heating 1kg water from 20°C to 80°C $\implies$ Use \underline{\hspace{2cm}} [N3]
        \item Melting 0.5kg ice at 0°C $\implies$ Use \underline{\hspace{2cm}} [N5]
        \item Cooling 2kg steam at 120°C to 110°C $\implies$ Use \underline{\hspace{2cm}} [N3]
        \item Boiling 0.2kg water at 100°C $\implies$ Use \underline{\hspace{2cm}} [N5]
    \end{itemize}
    \pause
    \textit{Answers: Q=mc$\Delta$T, Q=mL$_f$, Q=mc$\Delta$T, Q=mL$_v$}
    \vspace{1em}
    Today: Combining these when things mix and reach equilibrium.
\end{frame}

\section{Thermal Equilibrium Principle}
\begin{frame}{The Balancing Act [N2 Apply]}
    \frametitle{Principle of Thermal Equilibrium Problems}
    \textbf{Scenario:} Mix hot and cold substances in an \textit{isolated} system (no heat loss to surroundings).
    \vspace{1em}
    \textbf{Principle (Conservation of Energy):}
    \begin{itemize}
        \item Heat energy flows from hotter object(s) to colder object(s).
        \item Flow stops when thermal equilibrium (same final temperature $T_f$) is reached.
        \item Total heat energy \textbf{lost} by hot objects = Total heat energy \textbf{gained} by cold objects.
    \end{itemize}
    \vspace{1em}
    \begin{equation*}
    \alert{\sum Q_{lost} = \sum Q_{gained}}
    \end{equation*}
    \textit{Each 'Q' term could involve $mc\Delta T$ or $mL$ depending on temperature changes and phase changes.}
\end{frame}

\section{Guided Problem}
\begin{frame}{Solving an Equilibrium Problem [N2, N3 Apply]}
    \frametitle{Guided Example: Hot Metal in Cold Water}
    \textit{(Refer to Worksheet 3, Part 1 for step-by-step guidance)}
    \vspace{1em}
    \textbf{Problem:} 50g Copper ($c_{Cu} = 385$) at 90°C dropped into 100g Water ($c_{w} = 4186$) at 15°C. Find final temp ($T_f$).
    \vspace{1em}
    \textbf{Setup:}
    \begin{itemize}
        \item Identify Hot (Cu) / Cold (Water).
        \item $Q_{lost, Cu} = Q_{gained, w}$
        \item $(mc\Delta T)_{Cu} = (mc\Delta T)_{w}$
        \item $m_{Cu}c_{Cu}(T_{i,Cu} - T_f) = m_w c_w (T_f - T_{i,w})$
    \end{itemize}
    \vspace{1em}
    \textbf{Key Steps (Teacher demonstrates on board):}
    \begin{enumerate}
        \item Substitute values (ensure mass in kg if needed, though g cancels if consistent).
        \item Expand brackets carefully.
        \item Group $T_f$ terms on one side.
        \item Solve for $T_f$.
    \end{enumerate}
    \textit{[Numeracy focus: Algebraic manipulation]}
\end{frame}

\section{Efficiency Concept}
\begin{frame}{Energy Usefulness (Inquiry Q2/Q3)}
    \frametitle{Introduction to Thermal Efficiency}
    \textbf{Energy Transformations (Inquiry Q2):} Energy is conserved, but changes form.
    \vspace{1em}
    \textbf{Think about a Car Engine:}
    \begin{itemize}
        \item \textbf{Energy Input:} Chemical Energy in Fuel
        \item \textbf{Useful Energy Output:} Kinetic Energy (Motion)
        \item \textbf{Wasted Energy Output:} Heat (Exhaust, Friction, Engine Block heating -> lost via Conduction/Convection/Radiation [N4 link])
    \end{itemize}
    \vspace{1em}
    \textbf{Thermal Efficiency (Qualitative Definition):}
    \[ \text{Efficiency} = \frac{\text{Useful Energy Output}}{\text{Total Energy Input}} \]
    \begin{itemize}
        \item Always less than 1 (or 100\%). Why? Some energy is always "lost" as less useful thermal energy during transformations (related to the 2nd Law of Thermodynamics).
        \item \textbf{Relevance (Inquiry Q3):} Understanding losses helps improve efficiency (e.g., insulation, better engine design) $\implies$ Sustainability.
    \end{itemize}
\end{frame}

\section{Practice and Synthesis}
\begin{frame}{Practice Problems [N2, N3, N5 Apply]}
    \frametitle{Applying Equilibrium Principles}
    Now, try the problems in Worksheet 3, Part 2.
    \begin{itemize}
        \item Work individually or in pairs.
        \item Problem 1: Similar to guided example (metal in water).
        \item Problem 2: Challenge involving phase change (melting ice). Remember to include $Q=mL_f$ for the ice melting!
    \end{itemize}
    \vspace{1em}
    \textbf{Connecting back to Inquiry Questions:}
    \begin{itemize}
        \item Q1 (Temp/Energy/Motion): Underpins all calculations.
        \item Q2 (Transformation/Laws): Efficiency shows energy changes form, conservation applies.
        \item Q3 (Direction/Efficiency): Equilibrium determines direction, efficiency measures usefulness.
    \end{itemize}
\end{frame}

\section{Summary}
\begin{frame}{Lesson 3 Summary}
    \begin{itemize}
        \item Thermal equilibrium problems are solved using Conservation of Energy: $\sum Q_{lost} = \sum Q_{gained}$ [N2 Apply].
        \item These problems combine specific heat ($mc\Delta T$) [N3] and potentially latent heat ($mL$) [N5] calculations.
        \item Thermal efficiency describes how effectively input energy is converted to useful output energy [Inquiry Q3].
        \item Understanding heat transfer mechanisms [N4] is key to understanding energy losses and efficiency.
    \end{itemize}
    \vspace{1em}
    \textbf{Final Steps:}
    \begin{itemize}
        \item Complete Worksheet 3 Practice Problems.
        \item Complete \#MarkSense Quiz 3.
        \item Review all concepts from the 3 lessons.
    \end{itemize}
\end{frame}

\begin{frame}
    \centering
    \textbf{Thank you!}\\
    End of Thermodynamics Introduction. Questions?
\end{frame}

\end{document}
