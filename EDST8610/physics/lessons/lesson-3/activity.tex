\documentclass[11pt, a4paper]{article} % Using article class for simpler activity sheet
\usepackage{geometry}
\geometry{left=2cm, right=2cm, top=2cm, bottom=2cm}
\usepackage{graphicx}
\usepackage{hyperref}
\usepackage{array}
\usepackage[utf8]{inputenc}
\usepackage[T1]{fontenc}
\usepackage{amsmath}
\usepackage{siunitx}

\title{Year 11 Physics - Activity Sheet 3 \\ Thermal Equilibrium Calculations}
\date{Module 3 - Lesson 3}
\author{Thermodynamics}

\begin{document}
\maketitle

\section*{Aim}
To apply the principle of conservation of energy ($Q_{lost} = Q_{gained}$) to solve problems involving thermal equilibrium between substances, incorporating specific heat capacity and potentially latent heat.

\section*{Knowledge Nodes Targeted}
\begin{itemize}
    \item N2: Thermal Equilibrium (Applying the concept quantitatively)
    \item N3: Specific Heat (Used within equilibrium calculations)
    \item N5: Latent Heat (Used within equilibrium calculations involving phase change)
\end{itemize}

\section*{Activity: Guided and Practice Problem Solving}
This activity focuses on applying the concepts learned in Lessons 1 and 2 to quantitative problems. The main tool is the principle of energy conservation in an isolated system.

\subsection*{Core Principle}
In an isolated system where hotter and colder substances are mixed, heat energy will transfer from the hotter substance(s) to the colder substance(s) until thermal equilibrium is reached (i.e., they reach the same final temperature, $T_f$). The total energy lost by the initially hotter substance(s) must equal the total energy gained by the initially colder substance(s).
\[ \sum Q_{lost} = \sum Q_{gained} \]
Where Q can be calculated using $Q=mc\Delta T$ for temperature changes and $Q=mL$ for phase changes. Remember:
\begin{itemize}
    \item For heat loss: $\Delta T = T_{initial, hot} - T_{final}$
    \item For heat gain: $\Delta T = T_{final} - T_{initial, cold}$
    \item Phase change energy must be included if a substance melts/freezes or boils/condenses during the process.
\end{itemize}

\subsection*{Guided Problem (Refer to Worksheet 3 Part 1)}
The teacher will guide the class through solving the problem of mixing hot copper with cold water, demonstrating the setup and algebraic solution for the final equilibrium temperature ($T_f$).

\subsection*{Practice Problems (Refer to Worksheet 3 Part 2)}
Students will work individually or in pairs to solve the practice problems provided on the worksheet. These problems may involve:
\begin{itemize}
    \item Mixing two substances with no phase change (applying $Q=mc\Delta T$ on both sides).
    \item Mixing substances where one undergoes a phase change (applying $Q=mL$ and $Q=mc\Delta T$ as needed).
\end{itemize}

\subsection*{Required Data}
A data table with specific heat capacities (c) and latent heats (L) for relevant materials (e.g., water, ice, steam, copper, lead, aluminium) is required. (This should be provided with Worksheet 3 or displayed).

\section*{Numeracy Focus}
\begin{itemize}
    \item Setting up algebraic equations based on the energy conservation principle.
    \item Correctly identifying terms for heat loss and heat gain.
    \item Accurately substituting values (including unit consistency, e.g., mass in kg if 'c' or 'L' are per kg).
    \item Solving the resulting algebraic equations for the unknown variable (often $T_f$ or an unknown mass).
\end{itemize}

\section*{Literacy Focus}
\begin{itemize}
    \item Clearly justifying the steps taken in the problem-solving process.
    \item Explaining the meaning of the energy conservation equation in the context of the problem.
\end{itemize}

\end{document}
