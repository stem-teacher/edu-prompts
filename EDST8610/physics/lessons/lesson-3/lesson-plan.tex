\documentclass[11pt, a4paper]{article}
\usepackage{geometry}
\geometry{left=2cm, right=2cm, top=2cm, bottom=2cm}
\usepackage{hyperref}
\usepackage{array}
\usepackage[utf8]{inputenc}
\usepackage[T1]{fontenc}

\title{Year 11 Physics - Lesson Plan 3/3 \\ Thermodynamics: Equilibrium Calculations & Efficiency Concepts}
\date{Based on NSW Stage 6 Syllabus (Module 3)}
\author{Philip Haynes}

\begin{document}
\maketitle
\vspace{-2em}

\section*{Lesson Overview}
\begin{itemize}
    \item \textbf{Lesson Title:} Thermal Equilibrium Problems and Introduction to Efficiency
    \item \textbf{Duration:} 60 minutes
    \item \textbf{Focus Inquiry Question:} How are temperature, thermal energy, and particle motion related? (Application in equilibrium). What predicts and determines the direction and efficiency of energy transfer? (Equilibrium as predictor, Efficiency concept). How does energy transformation underpin the laws of thermodynamics? (Efficiency link to conservation/losses).
\end{itemize}

\section*{Syllabus Alignment & Knowledge Nodes Targeted}
\begin{itemize}
    \item \textbf{Outcomes:} PH11-10, PH11/12-6 (Problem Solving), PH11/12-7 (Communicate)
    \item \textbf{Content:} ACSPH022 (Application)
    \item \textbf{Knowledge Nodes:} N2 (Thermal Equilibrium - Analyse/Apply), integrated application of N3 (Specific Heat) and N5 (Latent Heat). Conceptual links to N4 (Transfer Mechanisms) and Inquiry Q2/Q3 (Efficiency/Transformation).
\end{itemize}

\section*{Student Learning Objectives (Aligned with Nodes)}
Students will be able to:
\begin{itemize}
    \item Apply the principle of conservation of energy (Heat Lost = Heat Gained) to solve quantitative thermal equilibrium problems, potentially involving phase changes (N2 Analyse/Apply, N3 Apply, N5 Apply).
    \item Explain thermal efficiency conceptually (useful energy output / total energy input) and relate it to energy losses via heat transfer mechanisms (Links N4, Inquiry Q3).
    \item Conceptually link energy transformations and losses to thermodynamic principles (Links Inquiry Q2).
    \item [Literacy] Justify the setup and steps in solving thermal equilibrium problems. Explain the concept of thermal efficiency and its relevance (N2, Inquiry Q3).
    \item [Numeracy] Set up and solve multi-step algebraic equations involving Q=mcΔT and Q=mL in equilibrium scenarios (N2, N3, N5).
\end{itemize}

\section*{Lesson Structure \& Activities}

\subsection*{Introduction (10 mins)}
\begin{itemize}
    \item \textbf{Teacher Activity:} Review key formulae Q=mcΔT (N3) and Q=mL (N5) from L2. Revisit Thermal Equilibrium concept (N2). Explicitly state the energy conservation principle for isolated systems: Heat Lost by hotter object(s) = Heat Gained by colder object(s). Write equation form on board: $\Sigma Q_{lost} = \Sigma Q_{gained}$.
    \item \textbf{Student Activity:} Recall formulae and equilibrium concept. Understand the energy balance equation setup for equilibrium problems.
    \item \textbf{Pedagogy Focus:} Retrieval Practice, Establishing the core principle for complex problem solving (Conservation of Energy).
\end{itemize}

\subsection*{Exploration (25 mins)}
\begin{itemize}
    \item \textbf{Teacher Activity:} Lead a Guided Problem-Solving session for a thermal equilibrium calculation (e.g., hot metal in cold water - see Activity Sheet 3 / W/S 3 Part 1). Emphasise identifying initial states, final equilibrium state (Tf), and setting up the $Q_{lost} = Q_{gained}$ equation using appropriate N3/N5 formulae for each substance. Introduce Efficiency concept (Inquiry Q3): Use car engine/power plant example. Define qualitatively (Useful Out / Total In). Discuss why heat loss (via N4 mechanisms) prevents 100% efficiency. Link briefly to energy conservation and transformation (Inquiry Q2 - energy changes form, some becomes less useful thermal energy). Discuss relevance (sustainability, technology).
    \item \textbf{Student Activity:} Follow guided problem steps on Worksheet 3 Part 1. Ask questions during the process. Participate in discussion on efficiency, identifying examples of energy input, useful output, and wasted heat output. [N2 Apply, N3 Apply]
    \item \textbf{Pedagogy Focus:} Guided Problem Solving (Cognitive Load Management for complex problems), Conceptual Breadth (Efficiency), Linking Concepts (N4 to Efficiency Losses), Contextualisation (Relevance).
    \item \textbf{Literacy Focus:} Explaining the setup of the conservation equation, defining efficiency.
\end{itemize}

\subsection*{Consolidation (25 mins)}
\begin{itemize}
    \item \textbf{Teacher Activity:} Assign 1-2 practice problems on Worksheet 3 (Part 2) involving thermal equilibrium, potentially including phase changes [N5 integration]. Encourage students to work in pairs (Collaborative Learning). Circulate, offering targeted support. Briefly review how the three lessons addressed the main Inquiry Questions. Distribute \#MarkSense Quiz 3.
    \item \textbf{Student Activity:} Work collaboratively or individually on practice problems (W/S 3 Part 2). Seek help if needed. Reflect on Inquiry Question connections. Complete \#MarkSense Quiz 3 (end of class or homework). [N2 Apply, N3 Apply, N5 Apply]
    \item \textbf{Pedagogy Focus:} Collaborative/Independent Practice, Application of multiple concepts (N2, N3, N5), Synthesis (Linking back to Inquiry Qs), Formative Assessment.
    \item \textbf{Numeracy Focus:} Solving multi-step equilibrium problems possibly involving phase change calculations (N2, N3, N5).
\end{itemize}

\section*{Resources Required}
\begin{itemize}
    \item Worksheet 3 (separate PDF).
    \item \#MarkSense Quiz 3 (included on Worksheet 3 PDF).
    \item Data table with relevant specific heat (c) and latent heat (L) values.
    \item Calculators.
    \item Projector/Whiteboard.
\end{itemize}

\section*{Assessment}
\begin{itemize}
    \item \textbf{Formative:} Teacher observation during collaborative problem-solving. Review of Worksheet 3 problem-solving approaches and answers. Analysis of \#MarkSense Quiz 3 results.
\end{itemize}

\section*{Differentiation}
\begin{itemize}
    \item \textbf{Support:} Provide pre-structured templates for setting up equilibrium equations ($Q_{lost}$ side = $Q_{gained}$ side). Focus practice on problems without phase changes initially.
    \item \textbf{Extension:} Include problems with multiple substances mixing, or where heat loss to the container (calorimeter) must be considered. Challenge students to research the thermodynamic efficiency of different types of power plants.
\end{itemize}

\end{document}
