\documentclass[11pt, a4paper]{article}
\usepackage{geometry}
\geometry{left=2cm, right=2cm, top=2cm, bottom=2cm}
\usepackage{hyperref}
\usepackage{array}
\usepackage[utf8]{inputenc}
\usepackage[T1]{fontenc}

\title{Year 11 Physics - Lesson Plan 1/3 \\ Thermodynamics: Particles, Temperature & Energy Flow}
\date{Based on NSW Stage 6 Syllabus (Module 3)}
\author{Philip Haynes}

\begin{document}
\maketitle
\vspace{-2em} % Reduce space after title

\section*{Lesson Overview}
\begin{itemize}
    \item \textbf{Lesson Title:} Thermodynamics: Relating Particles, Temperature, and Energy Transfer
    \item \textbf{Duration:} 60 minutes
    \item \textbf{Focus Inquiry Question:} How are temperature, thermal energy, and particle motion related? (Introduction to Q3: direction of energy transfer)
\end{itemize}

\section*{Syllabus Alignment & Knowledge Nodes Targeted}
\begin{itemize}
    \item \textbf{Outcomes:} PH11-10, PH11/12-3 (Conduct Invest.), PH11/12-7 (Communicate)
    \item \textbf{Content:} ACSPH018, ACSPH016, ACSPH022 (conceptual intro)
    \item \textbf{Knowledge Nodes:} N1 (Temp/KE Relation), N4 (Transfer Mechanisms), N2 (Thermal Equilibrium - Concept)
\end{itemize}

\section*{Student Learning Objectives (Aligned with Nodes)}
Students will be able to:
\begin{itemize}
    \item Explain the relationship between temperature and the average kinetic energy of particles (N1 - Understand).
    \item Identify and describe conduction, convection, and radiation with examples (N4 - Understand).
    \item Explain conduction in solids using the particle model (N4 - Understand).
    \item Define thermal equilibrium conceptually as no net energy transfer (N2 - Understand).
    \item Predict the direction of heat flow based on temperature differences (Links N1, N2, Inquiry Q3).
    \item [Literacy] Define temperature, thermal energy, conduction, convection, radiation, thermal equilibrium precisely (N1, N4, N2).
    \item [Numeracy] Qualitatively interpret particle energy distributions/visualisations (N1).
\end{itemize}

\section*{Lesson Structure \& Activities}

\subsection*{Introduction (10 mins)}
\begin{itemize}
    \item \textbf{Teacher Activity:} Display Inquiry Questions 1, 2, 3. State focus on Q1. Engage with prompt: "Metal vs wood chair feeling cold/warm". Facilitate brief discussion. Introduce Thermodynamics scope. Provide historical (Steam Engine) and future (Climate/IT) context. Define core terms on board/slide: Temperature (Avg KE), Thermal Energy (Total KE+PE), Heat (Transfer of TE). [N1 Definitions]
    \item \textbf{Student Activity:} Note Inquiry Questions. Participate in discussion. Record key definitions from board/slide (support via Worksheet 1).
    \item \textbf{Pedagogy Focus:} Contextualization (Motivation), Activate Prior Knowledge, Core Terminology (Literacy N1).
\end{itemize}

\subsection*{Exploration (30 mins)}
\begin{itemize}
    \item \textbf{Teacher Activity:} Conduct/Show Demo/Simulations (See Activity Sheet 1 for details):
        \begin{itemize}
            \item Conduction (N4): Heat metal rod, explain particle vibration transfer.
            \item Convection (N4): Show video/sim of convection currents, explain density changes.
            \item Radiation (N4): Use IR camera/heat lamp/sim, explain EM wave transfer. Link to user prompt: hot metal radiating heat and light.
        \end{itemize}
        Guide PhET Simulation "Energy Forms and Changes" exploration (link on Activity Sheet 1). Focus on particle view vs. temperature (N1). Introduce Equilibrium concept (N2) - what happens when hot/cold objects touch? Discuss direction of flow (Inquiry Q3 link).
    \item \textbf{Student Activity:} Observe demos/sims, explain using particle model (N1, N4). Use PhET simulation on laptops, guided by Worksheet 1 prompts. Discuss equilibrium concept. Complete relevant parts of Worksheet 1.
    \item \textbf{Pedagogy Focus:} Active Learning (Observation/Prediction), Multimodal Input (Demo/Sim), Visualising Microscopic Processes (N1, N4), Guided Inquiry (N2). Cognitive Science: Dual Coding, reducing load via visualisation.
    \item \textbf{ICT Integration:} PhET Simulation.
\end{itemize}

\subsection*{Consolidation (20 mins)}
\begin{itemize}
    \item \textbf{Teacher Activity:} Lead class discussion summarising N1, N4. Explicitly address metal/wood chair question using conduction/conductivity concept (N4). Reiterate equilibrium concept (N2) and direction of heat flow. Distribute Worksheet 1 for completion. Distribute \#MarkSense Quiz 1.
    \item \textbf{Student Activity:} Participate in discussion, complete Worksheet 1 (definitions, explanations for N1, N4, N2 concept). Complete \#MarkSense Quiz 1 (end of class or homework).
    \item \textbf{Pedagogy Focus:} Concept Consolidation, Linking Micro-Macro, Formative Assessment.
\end{itemize}

\section*{Resources Required}
\begin{itemize}
    \item Teacher demonstrations materials (metal rod, heat source, etc. - See Activity Sheet 1) OR Simulation/Video access.
    \item Student laptops with internet access.
    \item PhET Simulation links (on Activity Sheet 1).
    \item Worksheet 1 (separate PDF).
    \item \#MarkSense Quiz 1 (included on Worksheet 1 PDF).
    \item Projector/Whiteboard.
\end{itemize}

\section*{Assessment}
\begin{itemize}
    \item \textbf{Formative:} Teacher observation of student participation in discussions and simulation use. Review of Worksheet 1 responses. Analysis of \#MarkSense Quiz 1 results.
\end{itemize}

\section*{Differentiation}
\begin{itemize}
    \item \textbf{Support:} Provide sentence starters for explanations on worksheet. Pair students for simulation exploration. Pre-teach key vocabulary.
    \item \textbf{Extension:} Ask students to research specific thermal conductivity values and explain differences. Challenge students to explain convection in weather patterns.
\end{itemize}

\end{document}
