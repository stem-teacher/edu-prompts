\documentclass[11pt, a4paper]{article} % Using article class for simpler activity sheet
\usepackage{geometry}
\geometry{left=2cm, right=2cm, top=2cm, bottom=2cm}
\usepackage{graphicx}
\usepackage{hyperref}
\usepackage{array}
\usepackage[utf8]{inputenc}
\usepackage[T1]{fontenc}

\title{Year 11 Physics - Activity Sheet 1 \\ Demonstrating Heat Transfer Mechanisms}
\date{Module 3 - Lesson 1}
\author{Thermodynamics}

\begin{document}
\maketitle

\section*{Aim}
To observe and explain the three primary mechanisms of heat transfer: conduction, convection, and radiation, and relate these to particle motion and temperature changes.

\section*{Knowledge Nodes Targeted}
\begin{itemize}
    \item N1: Temp/KE Relation (Observing effect of heating)
    \item N4: Transfer Mechanisms (Identifying and explaining each mode)
    \item N2: Thermal Equilibrium (Conceptual introduction via heat flow direction)
\end{itemize}

\section*{Part A: Demonstrations (Teacher Led or Video/Simulation)}

\subsection*{Demo 1: Conduction}
\begin{itemize}
    \item \textbf{Materials:} Metal rod (e.g., copper or aluminium), heat source (Bunsen burner or torch), heat-resistant mat, (optional: thermal camera, wax dots along rod).
    \item \textbf{Procedure:}
        \begin{enumerate}
        \item Place the metal rod on the heat-resistant mat.
        \item Carefully heat ONE end of the rod with the heat source.
        \item Observe how the heat travels along the rod (e.g., using touch carefully away from heat, thermal camera, or melting wax dots).
        \end{enumerate}
    \item \textbf{Safety:} Wear safety glasses. Handle hot rod with tongs. Be aware of hot surfaces.
    \item \textbf{Observation Prompt (for Worksheet 1):} Describe how the energy transferred from the hot end to the cold end. Explain using the particle model (vibrations, collisions). [N4]
\end{itemize}

\subsection*{Demo 2: Convection}
\begin{itemize}
    \item \textbf{Materials:} Large beaker or flask of water, heat source, (optional: potassium permanganate crystal or food colouring, small paper pieces).
    \item \textbf{Procedure (Option 1 - Visualisation):}
        \begin{enumerate}
        \item Fill beaker with cold water. Carefully drop a small crystal of KMnO4 or a drop of food colouring to the bottom near one side.
        \item Gently heat the water directly below the colourant.
        \item Observe the movement of the coloured water.
        \end{enumerate}
    \item \textbf{Procedure (Option 2 - Simulation):} Use PhET "States of Matter" or search for "convection current simulation/video". Observe fluid movement patterns when heated from below.
    \item \textbf{Observation Prompt:} Describe the pattern of movement observed. Explain why the fluid moves in this way, relating it to temperature, density, and particle movement. [N4]
\end{itemize}

\subsection*{Demo 3: Radiation}
\begin{itemize}
    \item \textbf{Materials:} Heat lamp or incandescent bulb, hand or thermometer, (optional: infrared thermometer/camera).
    \item \textbf{Procedure:}
        \begin{enumerate}
        \item Turn on the heat lamp/bulb.
        \item Carefully place a hand near (but not touching) the lamp. Feel the warmth.
        \item (Optional) Measure temperature near the lamp with IR thermometer or observe with IR camera.
        \end{enumerate}
    \item \textbf{Observation Prompt:} How did the heat reach your hand without direct contact or air movement being the main factor? What type of energy transfer is this? Does it require a medium? [N4]
\end{itemize}

\section*{Part B: PhET Simulation Exploration}

\subsection*{Simulation: Energy Forms and Changes}
\begin{itemize}
    \item \textbf{Link:} \href{https://phet.colorado.edu/en/simulations/energy-forms-and-changes}{https://phet.colorado.edu/en/simulations/energy-forms-and-changes}
    \item \textbf{Procedure:}
        \begin{enumerate}
        \item Open the simulation and select the "Intro" screen.
        \item Place a thermometer on the brick and the water.
        \item Place the brick and water on the stands.
        \item Check the "Energy Symbols" box.
        \item Use the slider to add heat to both the brick and the water.
        \item Observe:
            \begin{itemize}
                \item The movement/vibration of the particles (atoms/molecules) within the brick and water.
                \item The change in the temperature reading on the thermometers.
                \item The flow of 'E' energy symbols representing heat transfer.
            \end{itemize}
        \end{enumerate}
    \item \textbf{Observation Prompts (for Worksheet 1):}
        \begin{itemize}
            \item How did particle motion change when heat was added? [N1]
            \item How did temperature change? Relate particle motion to temperature. [N1]
        \end{itemize}
\end{itemize}

\end{document}
