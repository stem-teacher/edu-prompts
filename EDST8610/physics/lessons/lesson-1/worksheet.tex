\documentclass[11pt, a4paper]{article}
\usepackage{geometry}
\geometry{left=2cm, right=2cm, top=2cm, bottom=2cm}
\usepackage{amsmath} % For math
\usepackage{graphicx} % For images if needed
\usepackage{array}
\usepackage[utf8]{inputenc}
\usepackage[T1]{fontenc}
\usepackage{hyperref}

\title{Year 11 Physics - Worksheet 1 \\ Thermodynamics: Particles, Temperature & Energy Flow}
\date{Module 3}
\author{Student Name: \underline{\hspace{5cm}} ID: \underline{\hspace{3cm}}} % Placeholder for student info

\begin{document}
\maketitle

\section*{Part 1: Defining Concepts (Knowledge Nodes N1, N4, N2)}

1.  \textbf{Define} the following terms precisely using your understanding from the lesson:
    \begin{itemize}
        \item Thermodynamics: \vspace{1cm}
        \item Temperature (in terms of particle motion): \vspace{1cm}
        \item Thermal Energy: \vspace{1cm}
        \item Heat: \vspace{1cm}
        \item Conduction: \vspace{1cm}
        \item Convection: \vspace{1cm}
        \item Radiation (thermal): \vspace{1cm}
        \item Thermal Equilibrium: \vspace{1cm}
    \end{itemize}
    \textit{[Literacy Focus: Precise scientific terminology - N1, N4, N2]}

2.  Using the particle model, \textbf{explain} why a metal spoon left in hot soup quickly becomes hot, while a wooden spoon takes much longer. Mention the key heat transfer mechanism involved. [N4 Understand]
    \vspace{2cm}

3.  Give one real-world \textbf{example} for each type of heat transfer where it is the *primary* mode of transfer:
    \begin{itemize}
        \item Conduction Example:
        \item Convection Example:
        \item Radiation Example:
    \end{itemize}
    [N4 Understand]

\section*{Part 2: Observations & Explanations (Knowledge Nodes N1, N4)}

4.  From the PhET Simulation ("Energy Forms and Changes"):
    \begin{itemize}
        \item Describe what happened to the \textbf{motion} of the water/brick particles when heat energy was added. [N1 Understand] \vspace{1cm}
        \item What happened to the \textbf{temperature} reading as heat was added? [N1 Understand] \vspace{1cm}
        \item What is the relationship between the average kinetic energy of the particles and the temperature of the substance? [N1 Understand] \vspace{1cm}
        \textit{[Numeracy Focus: Qualitative interpretation of simulation visuals - N1]}
    \end{itemize}

5.  Consider the demonstrations of heat transfer:
    \begin{itemize}
        \item How does energy transfer differ fundamentally between conduction (e.g., metal rod) and radiation (e.g., heat lamp)? [N4 Understand] \vspace{2cm}
    \end{itemize}


\hrulefill
\section*{\#MarkSense Quiz 1}
\textbf{Instructions:} Choose the best answer for multiple choice questions. Write brief answers for short answer questions in the space provided.

\vspace{0.5cm}
\textbf{Student Name:} \underline{\hspace{5cm}} \textbf{ID:} \underline{\hspace{3cm}}
\vspace{0.5cm}

1.  Temperature is a measure of the \_\_\_\_ kinetic energy of particles in a substance. [N1]
    \begin{itemize}
        \item[A.] Total
        \item[B.] Average
        \item[C.] Potential
        \item[D.] Rotational
    \end{itemize}
    \textbf{Answer:} \underline{\hspace{1cm}}

2.  Heat transfer through the movement of fluids (liquids/gases) is primarily: [N4]
    \begin{itemize}
        \item[A.] Conduction
        \item[B.] Convection
        \item[C.] Radiation
        \item[D.] Advection
    \end{itemize}
    \textbf{Answer:} \underline{\hspace{1cm}}

3.  Explain why putting a lid on a hot cup of coffee keeps it warm longer, mentioning at least TWO heat transfer mechanisms. [N4, N2 conceptual link] (2 marks)
    \vspace{3cm}


\end{document}
